\documentclass{article}

\usepackage{amsfonts}
\usepackage[all]{xy}
\usepackage{amssymb}
\usepackage{amsmath}
\usepackage{mathrsfs}
\usepackage{amsthm}
\usepackage{enumerate}
\usepackage[hidelinks]{hyperref}
\usepackage{tikz}  

\usepackage{geometry}
\geometry{a4paper,left=2cm,right=2cm,top=2cm,bottom=2cm}

\newtheorem{definition}{Definition}[section]
\newtheorem{proposition}[definition]{Proposition}
\newtheorem{lemma}[definition]{Lemma}
\newtheorem{theorem}[definition]{Theorem}
\newtheorem{corollary}[definition]{Corollary}
\newtheorem{remark}[definition]{Remark}
\newtheorem{fact}[definition]{Fact}
\newtheorem{assertion}[definition]{Assertion}
\newtheorem{example}[definition]{Example}
\newtheorem{problem}{Problem}
\newtheorem*{ques}{Question}
\setcounter{section}{0}

\title{Sarkisov program for CY}
\author{wyz}
%\date{\today}

\begin{document}
\maketitle
%\tableofcontents
%\newpage
\section{Intro}
One of the goal of birational geometry is to classify varities up to birational equivalence. MMP is a way to find primitive for each class. But the output of MMP may not be unique.Therefore a nature question is the relationship between different primitives.
In another way, assume different MMP on $ X $ ends with different outputs $ X_{1},X_{2} $, and let $ \phi:X_{1} \dashrightarrow X_{2} $ be the induced birational map. Then how to descripe $ \phi $?

For minimal model, $ \phi $ is composition of flops. For Mori fibre space, $ \phi $ is composition of Sarkisov links.

\subsection{Further}

\section{Review}

HM13 shows SP for klt, by partition of $ \mathcal{E}_{A}(V) $.

There is another description by divisorial ring.

\section{Overview}

\end{document}
